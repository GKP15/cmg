\documentclass[11pt,a4paper]{article}
%\usepackage{beamerarticle}

%\usefonttheme[onlymath]{serif}
\usepackage[ngerman]{babel}
\usepackage[utf8]{inputenc}
\usepackage[T1]{fontenc}
\usepackage{tikz}
\usetikzlibrary{positioning, arrows}
\usepackage{listings}
\usepackage{fancybox}
\usepackage{color}
\usepackage{hyperref}
\usepackage{fancyhdr}

\pagestyle{fancy}
\lhead{\today}
\rhead{Formatiert von: PP}
\chead{Gruppe: swp15.gkp}
%\cfoot{center of the footer!}
\renewcommand{\headrulewidth}{0.8pt}
%\renewcommand{\footrulewidth}{0.4pt}

\begin{document}
\center \large Universität Leipzig - Softwaretechnik Praktikum 2014/2015\\
\center \Huge Glossar \\
\par\bigskip

\small zum Projekt: Ein kartenbasiertes “Multiplayer”-Spiel

\par\bigskip

\tableofcontents 

\clearpage

\flushleft
\section{Begriffe}
\subsection{Geodaten}  
Digitale Informationen, denen auf der Erdoberfläche eine bestimmte räumliche Lage zugewiesen werden kann. Von besonderer Bedeutung für Geodaten sind Metadaten, die die eigentlichen räumlichen Daten zum Beispiel hinsichtlich eines Zeitbezugs oder der Entstehung beschreiben.
\subsection{Metadaten}
Metadaten enthalten Informationen über andere Daten, aber nicht die Daten selbst.
\subsection{Linked Open Data}
Linked Open Data (LOD) bezeichnet im World Wide Web frei verfügbare Daten, die per Uniform Resource Identifier (URI) identifiziert sind und darüber direkt per HTTP abgerufen werden können und ebenfalls per URI auf andere Daten verweisen. Idealerweise werden zur Kodierung und Verlinkung der Daten das Resource Description Framework (RDF) und darauf aufbauende Standards wie SPARQL und die Web Ontology Language (OWL) verwendet, so dass Linked Open Data gleichzeitig Teil des Semantic Web ist. Die miteinander verknüpften Daten ergeben ein weltweites Netz, das auch als „Linked [Open] Data Cloud“ oder „Giant Global Graph“ bezeichnet wird. Dort wo der Schwerpunkt weniger auf der freien Nutzbarkeit der Daten wie bei freien Inhalten liegt (Open Data), ist auch die Bezeichnung \textbf{Linked Data} üblich.\subsection{LinkedGeoData}
Bezeichnet die Verknüpfung von Geodaten als Linked Data.

\subsection{Abfragesprachen}
Abfragesprachen sind formale Sprachen zur Informationssuche. Eine  Abfrage (Query) liefert einen Teil der zugrundeliegenden Informationen, filtert diese also.



\subsection{Tripel}
Ein Tripel ist ein 3-Tupel. In unserem Zusammenhang beschäftigen wir uns mit Tripel, bestehend aus einem Subjekt, einem Prädikat und einem Objekt. Als Relationen werden URIs verwendet.
\subsection{URI}
Ein URI, besteht aus einer  Zeichenfolge und dient zu Identifizierung von abstrakten und physischen  Ressourcen. Dazu gehören Webseiten, Dateien und Email-Empfänger.

\subsection{API}
Eine API, eine  Anwendungsprogrammierschnittstelle, ist ein von einem Softwaresystem zur  Verfügung gestellter Programmteil, um anderen Programmen die  Verknüpfung mit dem System zu ermöglichen.
\subsection{Datenbankensystem}
Ein Datenbanksystem ist ein  System zur dauerhaften, effizienten und widerspruchsfreien Speicherung  und Verwaltung großer Datenmengen.
\subsection{OpenStreetMap}
OpenStreetMap ist eine  Datenbank mit geographischen Daten, welche unter der Open Database License verwendet werden dürfen. Eine Nutzung durch Webseiten und  Anwendung ist also unbeschränkt und unentgeltlich möglich.
\subsection{Ontologie}
Ontologien in der Informatik  sind meist sprachlich gefasste und formal geordnete Darstellungen einer  Menge von Begrifflichkeiten und der zwischen ihnen bestehenden  Beziehungen in einem bestimmten Gegenstandsbereich. Sie werden dazu  genutzt, „Wissen“ in digitalisierter und formaler Form zwischen Anwendungsprogrammen  und Diensten auszutauschen. Wissen umfasst dabei sowohl Allgemeinwissen  als auch Wissen über sehr spezielle Themengebiete und Vorgänge.
\subsection{Graph}
Ein Graph ist eine abstrakte Struktur, die eine Menge von Objekten zusammen mit den zwischen diesen Objekten bestehenden Verbindungen repräsentiert.

\subsection{Camel-Case}
Schreibweise für Variablen, die vorschreibt, dass ausschliesslich Kleinbuchstaben verwendet werden, und alle dem ersten Wort folgenden Wörter mit einem Großbuchstaben beginnen sollen. \\
        Beispiele: camelCaseSchreibweise, variablenName

\subsection{Codechecker}
 Programm welches die Einhaltung bestimmter, definierbarer Codeconventions überwachen kann.

\subsection{Codeconvention}
Menge von Regeln, die bezüglich des Programmcodes eingehalten werden sollen zb Tabulatorlaenge, Variablenbezeichner, Klammernverwendung, Einrückungen etc.

\subsection{Codingstandards}
siehe Codeconvention

\subsection{Entwicklungsumgebung}
Eine Entwicklungsumgebung ist eine Sammlung von Programmen, mit denen Softwareentwicklung bearbeitet werden kann. Verfügt in der Regel mindestens über die Komponenten Texteditor(mit Syntax-Highlighter), Compiler und Debugger.

\subsection{Compiler}
Ein Compiler wandeln  den menschenlesbaren Programmiercode in maschinenlesbaren  Programmiercode um.

\subsection{Git}
Web-basierte Anwendung zum Speichern, Synchronisieren und zur Versionskontrolle von Quelltext.

\subsection{IDE} 
IDE steht für Entwicklungsumgebung, ausgeschrieben: Integrated Development Environment.

\subsection{Plattformunabhängig} Damit ist im Informatikkontext gemeint, dass ein Programm unabhängig von seinem Wirts-Betriebssystem lauffähig ist.

\subsection{Programmcode} Programmcode (auch Programmiercode) bezeichnet die menschenlesbare, compilierbare Menge von Anweisungen, die nach dem Compilieren für die Maschine ausführbar sind.

\subsection{Programmierstandards}
siehe Codeconvention

\subsection{Quelltext}
siehe Programmcode

\subsection{GUI(graphical user interface)} Ist eine graphische Benutzeroberfläche mit derer Hilfe ein Anwender in einem Programm agieren kann.

\subsection{JavaScript} Ist eine Scriptbasierte Programmiersprache

\subsection{Framework}
Ein Framework stellt ein Programmiergerüst dar, nach dessen Vorgaben ein Programmierer eine Anwendung erstellt.

\subsection{JUnit} Ein Framework zum Testen von Java-Programmen das besonders für automatisierte Tests einzelner Module geeignet ist.

\subsection{JSUnit} Ein Framework zum Testen von JavaScript-Programmen.


\section{Konzepte}
\subsection{Sematisches Web}
Das Semantische Web (engl. Semantic Web) ist ein Konzept bei der Entwicklung des World Wide Webs und des   Internets. Im Rahmen zur Weiterentwicklung zum Internet der Dinge wird es erforderlich, dass Maschinen die von   Menschen  zusammengetragenen Informationen verarbeiten können. All die   in  menschlicher Sprache ausgedrückten Informationen im Internet sollen   mit  einer eindeutigen Beschreibung ihrer Bedeutung (Semantik)   versehen  werden, die auch von Computern verstanden oder zumindest   verarbeitet  werden kann. Die maschinelle Verwendung der Daten aus dem   von Menschen  geflochtenen Netz der Daten ist nur möglich, wenn die   Maschinen deren  Bedeutung eindeutig zuordnen können; nur dann stellen   sie Informationen  dar.

\subsection{Graphdatenbanken}
Eine Graphdatenbank (oder graphenorientierte Datenbank) ist eine Datenbank, die Graphen benutzt, um stark vernetzte Informationen darzustellen und abzuspeichern. Ein solcher Graph besteht aus Knoten und Kanten, den Verbindungen zwischen den Knoten. Sowohl Knoten als auch Kanten können Eigenschaften, sogenannte Properties (bspw. Gewicht: 10kg, Farbe: Rot, Name: Alice), besitzen. In diesem Zusammenhang spricht man folglich auch von einem Property-Graphen. Durch diese Spezialisierung auf Property-Graphen unterscheiden sich Graphdatenbanken von den klassischen Datenmodellen der Relationalen Datenbanken aber auch der RDF/Triple-/Quad-Stores, welche vorwiegend im Semantic Web Anwendung finden.
\subsection{RDF}
RDF steht für \textbf{Resource Description Framework} (Repräsentationssprache für Ressourcen) und stellt eine
Form von Linked Data dar. Zur Beschreibung der Ressourcen und deren Relationen wird ein 
graphenbasiertes Datenmodell genutzt. RDF-Daten können unterschiedlich repräsentiert werden.
Einige Beispiele: RDF/XML, Notation-3 (N3), Turtle, N-Triples, RDFa, und RDF/JSON.
\subsection{FOAF (Friend of a Friend)}
Dieses  Projekt ist zur maschinenlesbaren Modellierung sozialer Netzwerke  entwickelt worden. Mithilfe eines RDF-Schemas werden Klassen und  Eigenschaften definiert, die in einem XML-basierten RDF-Dokument  verwendet werden können. FOAF ist eine der ersten Anwendungen von  Semantic-Web-Technologien.
Für dieses Projekt ist ein Teil des Vokabulars als Relation in den Tripel von Interesse.
\subsection{SPARQL}
\textbf{S}PARQL \textbf{P}rotocol \textbf{A}nd \textbf{R}DF \textbf{Q}uery \textbf{L}anguage, ist eine auf Graphen basierende Abfragesprache für RDF.
\subsection{OWL}
OWL ist eine formale Beschreibungssprache und steht für Web Ontology Language. OWL basiert auf der RDF Syntax. Es erweitert RDF um die Ausdrucksmächtigkeit zu verändern, meist wird sie erheblich erweitert. OWL besteht aus Klassen, Instanzen und Eigenschaften. Klassen können Eigenschaften besitzen. Einzelne Individuen dieser Klassen sind die Instanzen.

\subsection{XML}
Die \textbf{Extensible Markup Language} (engl. „erweiterbare Auszeichnungssprache“), abgekürzt XML, ist eine Auszeichnungssprache zur Darstellung hierarchisch strukturierter Daten in Form von Textdateien. XML wird u. a. für den plattform- und implementationsunabhängigen Austausch von Daten zwischen Computersystemen eingesetzt, insbesondere über das Internet.

\section{Aspekte}
\subsection{Einleitung}
Bei dem Projekt GeoKnowpoly geht es darum aus bereitsvorhandenem Kartenmaterial einen Spielplan zu erstellen. Außerdem sollen auch Realdaten entweder nur in den Spielplan oder auch in den Spielverlauf mit einfließen. Das Spielprinzip ist noch völlig offen.
\subsection{Ziel}
Unser Ziel ist es mit Hilfe von Karten- und Realdaten ein multiplayerbasiertes Spiel zu planen, entwerfen und realisieren.
Dies soll möglichst Simpel und Unterhaltsam ausfallen und vor allem, zum Projektende, zu 100\% funktionsfähig sein. 
\subsection{Vergleichbare Projekte}
Vergleichbare Projekte:
MapsTD ist ein Towerdefens-Spiel entwickelt von Duncan Barclay. Das Spiel basiert auf der Google Maps API, Mootools, Page visibility code by David Walsh, Watercolour tiles by Stamen Design (using OpenStreetMap data).
https://www.mapstd.com/
Außerdem gibt es noch eine Abwandlung des Spieles Risiko, diese befindet sich aber noch in der Entwicklung, so müssen zum Beispiel alle Mitspieler an einem Computer spielen.
http://www.ashotoforangejuice.com/gskirm/
Die meisten anderen Projekte, die die Google Maps API nutzen sind Wissensspiele bei denen man geografische Fragen beanworten muss wie zum Beispiel:
http://smartypins.withgoogle.com/
\subsection{Probleme}
Es gibt kaum ähnliche Projekte, die die Maps API zum erstellen von Spielkarten benutzen somit wird es schwer Lösungen für sehr Spielspezifiche Fragen recherchieren zu können. Desweiteren wird es schwer ein Spielprinzip zu finden, welches uns weder unter noch überfordert, beziehungsweise ein Spielprinzip was sich leicht erweitern oder verringern lässt.
\section{Quellen}
\begin{itemize}
\item \url{http://de.wikipedia.org/wiki/Geodaten}
\item \url{http://linkeddatabook.com/editions/1.0/}
\item \url{http://linkeddata.org/faq}
\item \url{http://en.wikipedia.org/wiki/Linked_data}
\item \url{https://de.wikipedia.org/wiki/Abfragesprache}
\item \url{http://en.wikipedia.org/wiki/FOAF}
\item \url{http://de.wikipedia.org/wiki/Web_Ontology_Language}
\item \url{http://de.wikipedia.org/wiki/Programmierschnittstelle}
\item \url{http://de.wikipedia.org/wiki/Datenbank}
\item \url{http://de.wikipedia.org/wiki/OpenStreetMap}
\item \url{http://www.openstreetmap.de/faq.html}
\item \url{http://de.wikipedia.org/wiki/Uniform_Resource_Identifier}
\item \url{http://de.wikipedia.org/wiki/Metadaten}
\item \url{http://de.wikipedia.org/wiki/Linked_Open_Data}
\item \url{http://de.wikipedia.org/wiki/Graphdatenbank}
\item \url{http://de.wikipedia.org/wiki/Binnenmajuskel}
\item \url{http://en.wikipedia.org/wiki/Coding_conventions}
\item \url{http://en.wikipedia.org/wiki/Integrated_development_environment}
\item \url{http://en.wikipedia.org/wiki/Compiler}
\item \url{http://en.wikipedia.org/wiki/Git_%28software%29}
\item \url{http://git-scm.com/}
\item \url{http://de.wikipedia.org/wiki/Grafische_Benutzeroberfl%C3%A4che}
\item \url{http://de.wikipedia.org/wiki/JavaScript}
\item \url{http://de.wikipedia.org/wiki/Framework}
\item \url{junit.org/}
\item \url{jsunit.net/}
\end{itemize}
\end{document}