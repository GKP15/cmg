\documentclass[11pt,a4paper]{article}

\usepackage[ngerman]{babel}
\usepackage[utf8]{inputenc}
\usepackage{fancyhdr}

\pagestyle{empty}

\begin{document}
\flushleft
\section*{Aufgabe 1 \& 2}
\textit{Quellcode: Aufgabe\_1\_2.java}

Ausgabe im Anhang.

Es lässt sich erkennen, dass die Exponentialsdarstellung eine Abweichung im Bereich von $10^{-17}$ für alle x ergibt. Zur Verbesserung gibt es zwei weitere Verfahren: Die Berechnung über die Produkt- oder Reihenentwicklung.
Dabei erkennt man, dass die Produktentwicklung für $x \le 10^{-9} $ mit der in Java verfügbaren Routine übereinstimmt, jedoch vorher teilweise größere Fehler als die Exponentialdarstellung besitzt. Die Reihenentwicklung besitzt dagegen eine relativ hohe Genauigkeit, ihr größter Fehler liegt bei $2*10^{-19}$ für $x=10^{-3}$ und sie weicht auch sonst nur für zwei weitere x-Werte von der vorgegebenen Routine ab.

\section*{Aufgabe 3 (c)}
\textit{Quellcode: Aufgabe3.java}

Ausgabe im Anhang.

Bei der Vorwärtsrekursion treten für $k\ge 18$ bei geraden k negative Werte auf und der Betrag von $I_k$ steigt schnell an bis ca. $4*10^{17}$. Hier liegt offenbar ein Rechenfehler vor, da das Integral nach (a) monoton fallend und immer positiv ist.
Dieser Rechenfehler entsteht durch Fehlerfortpflanzung, da der Startwert $I_0=e -1$ schon nicht exakt in Maschinenzahlen dargestellt werden kann. Diese Ungenaugigkeit wird in jedem Rekursionsschritt verstärkt und es kommt zu der beobachteten Abweichung.
Die Rückwärtsrekursion nimmt als Startwert $I_63=0$ und umgeht so einen anfänglichen Fehler. Dadurch sind dies Werte für alle k auch nah an exakten Werten, bzw. den durch MATLAB berechneten.

\end{document}