\documentclass[11pt,a4paper]{article}
%\usepackage{beamerarticle}

%\usefonttheme[onlymath]{serif}
\usepackage[ngerman]{babel}
\usepackage[utf8]{inputenc}
\usepackage[T1]{fontenc}
\usepackage{tikz}
\usetikzlibrary{positioning, arrows}
\usepackage{listings}
\usepackage{fancybox}
\usepackage{color}
\usepackage{hyperref}
\usepackage{fancyhdr}

\pagestyle{fancy}
\lhead{\today}
\rhead{Author: Fabian Tronicke}
\chead{Gruppe: swp15.gkp}
%\cfoot{center of the footer!}
\renewcommand{\headrulewidth}{0.8pt}
%\renewcommand{\footrulewidth}{0.4pt}

\begin{document}
\center \large Universität Leipzig - Softwaretechnik Praktikum 2014/2015\\
\center \Huge Entwurfsbeschreibung zum Vorprojekt \\
\par\bigskip

\small zum Projekt: Ein kartenbasiertes “Multiplayer”-Spiel

\par\bigskip

\tableofcontents

\clearpage

\flushleft
\section{Allgemeines}
Im Vorprojekt wird eine kleine Web Applikation erstellt, welche die Architektur minimal implementiert und es ermöglicht die Spielfigur (den Pucman) mit den Pfeiltasten über eine beliebige Karte zu navigieren.
Sobald die Applikation geladen ist, hat man die Möglichkeit, einen Ort auszuwählen an dem man das Spiel beginnnen will.
Hat man den Ort ausgewählt, kommt eine weitere Spielfigur (der Geist) in Spiel, die sich zufallsgeneriert über die Karte bewegt. 
Man kann den Pucman nun mit a-w-d-s steuern und wenn man es schafft mit Pucman den Geist einzugfangen, wird ein Geräusch ausgelöst.
Das ganze Spielgeschehen ist durch einen Soundtrack unterlegt, der eine funktionale und inhaltliche Verbindung zwischen Bild und Musik generiert.
\clearpage
\section{Produktübersicht}
\clearpage
\section{Grundsätzliche Struktur- und Entwurfsprinzipien}
\clearpage
\section{Struktur- und Entwurfsprinzipien der einzelnen Pakete}
\clearpage
\section{Datenmodell}
\clearpage
\section{Testkonzept}
\clearpage
\section{Glossar}
\clearpage
\end{document}