\documentclass[11pt,a4paper]{article}
%\usepackage{beamerarticle}

%\usefonttheme[onlymath]{serif}
\usepackage[ngerman]{babel}
\usepackage[utf8]{inputenc}
\usepackage[T1]{fontenc}
\usepackage{tikz}
\usetikzlibrary{positioning, arrows}
\usepackage{listings}
\usepackage{fancybox}
\usepackage{color}
\usepackage{hyperref}
\usepackage{fancyhdr}

\pagestyle{fancy}
\lhead{\today}
\author{Gruppe: SWT15-GKP}
\rhead{Author: KV}
\chead{Gruppe: SWT15-GKP}
%\cfoot{center of the footer!}
\renewcommand{\headrulewidth}{0.8pt}
%\renewcommand{\footrulewidth}{0.4pt}
\title{Universität Leipzig - Softwaretechnik Praktikum 2014/2015 \\  Darstellung des Arbeitsstandes \\ zum Projekt: Ein kartenbasiertes “Multiplayer”-Spiel}

\begin{document}
\maketitle

\clearpage

\flushleft


\section{Anforderungspaket Vorprojekt (Ist-Zustand)}
 /LF11/ Als Spieler möchte ich entscheiden können an welchem Ort ich \noindent\hspace*{13mm} spielen möchte.\\
 /LF14/ Als Spieler möchte ich das Spielgeschehen sehen.\\
 /LF15/ Als Spieler möchte ich über verbleibende Leben und Punkte per \noindent\hspace*{13mm} Headup informiert werden.\\
 /LF17/ Als Spieler möchte ich sowohl Hintergrundmusik als auch \noindent\hspace*{13mm} Soundeffekte hören.\\
 /LF52/ Als Spieler will ich, dass Pacman angezeigt und animiert wird.\\
 /LF21/ Als Spieler möchte ich Pacman mit der Tastatur steuern können.\\
 /LF23/ Als Akteur möchte ich mich vor dem Spiel über die Karte bewegen \noindent\hspace*{13mm} können.\\
 /LF51/ Als Spieler möchte ich, dass während des Spielens die Karte nicht \noindent\hspace*{13mm} mehr verändert werden kann.


\section{Anforderungspaket Interface 12}
/LF53/ Als Spieler will ich, dass die Geister angezeigt und animiert \noindent\hspace*{13mm} werden. \textbf{8}\\
/LF54/ Als Spieler will ich, dass die Powerups angezeigt werden. \textbf{4}
\section{Anforderungspaket Gamedesign 30}
/LF20/ Als Spieler möchte ich Powerups aufsammeln können. \textbf{8}\\
/LF22/ Als Spieler möchte ich, dass die Geister verschiedene Strategien \noindent\hspace*{13mm} verfolgen. \textbf{12}\\
/LF24/ Als Spieler möchte ich verschiedene Powerups. \textbf{10}\\
\section{Anforderungspaket Highscores 16}
/LF30/ Als Spieler möchte ich sehen wer auf welcher Karte welche \noindent\hspace*{13mm} Highscores erzielt hat. \textbf{16}
\section{Anforderungspaket Mapcreation 42}
/LF40/ Als Spieler möchte ich, dass semantische Daten Einfluss auf die \noindent\hspace*{13mm} Kartengenese haben. \textbf{20}\\
/LF41/ Als Spieler möchte ich, dass die Levelerstellung deterministisch \noindent\hspace*{13mm} abläuft. \textbf{18}\\
/LF43/ Als Betreuer möchte ich, dass die Kartengenese als Modul \noindent\hspace*{13mm} ausgeführt ist. \textbf{4}
\section{Optional}
/LF32/ Als Spieler möchte ich meine Highscores auf Social Media posten.\\
/LF44/ Als Spieler möchte ich an einen beliebigen Ort spielen können.\\
/LF42/ Als Spieler möchte ich, dass es zu jedem Ort mehrere Levels gibt.\\
/LF31/ Als Spieler möchte ich mich auf der Seite einloggen um meine \noindent\hspace*{13mm} Highscores zu loggen.
\end{document}