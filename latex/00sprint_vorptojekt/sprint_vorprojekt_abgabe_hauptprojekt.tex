\documentclass{article}

\usepackage[ngerman]{babel}
\usepackage[utf8]{inputenc}
\usepackage[T1]{fontenc}
\usepackage{tikz}
\usetikzlibrary{positioning, arrows}
\usepackage{listings}
\usepackage{fancybox}
\usepackage{fancyhdr}
\usepackage{lastpage}
\usepackage{verbatim}
\usepackage{ulem}
\usepackage{float}
\usepackage{hyperref}

\title{Zettel4}
\pagestyle{fancy}
\fancyhf{}

%\usepackage{geometry}
%\geometry{a4paper,left=40mm,right=30mm, top=2cm, bottom=4cm} 


\lhead{Seite:\thepage / \pageref{LastPage}}
\rhead{Gruppe: swp15.gkp}
\chead{Zettel 4}

\renewcommand{\headrulewidth}{0.4pt}
%\renewcommand{\footrulewidth}{0.4pt}

\author{swp15.gkp}
\date{\today{}}


\begin{document}
\center { \huge \textbf{Das Hauptprojekt}} \\ 

\flushleft 
\normalsize
\tableofcontents

\section{Allgemeines}

\section{Produktübersicht}
Beschreibung der äußerlichen Funktionsmerkmale des Systems

\section{Grundsätzliche Struktur- und Entwurfsprinzipien}
Was sollte eine Informatikerin über das Gesamtsystem wissen, ehe sie sich Details zuwendet?

\section{Struktur- und Entwurfsprinzipien einzelner Pakete}
Was sollte ein Informatiker über 3. hinaus wissen, ehe er sich Details eines speziellen Pakets zuwendet?

\section{Datenmodell}

\section{Testkonzept}

\section{Glossar}

\end{document}