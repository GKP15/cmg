\documentclass[11pt,a4paper]{article}
%\usepackage{beamerarticle}

%\usefonttheme[onlymath]{serif}
\usepackage[ngerman]{babel}
\usepackage[utf8]{inputenc}
\usepackage[T1]{fontenc}
\usepackage{tikz}
\usetikzlibrary{positioning, arrows}
\usepackage{listings}
\usepackage{fancybox}
\usepackage{color}
\usepackage{hyperref}
\usepackage{fancyhdr}

\pagestyle{fancy}
\lhead{\today}
\rhead{Author: Fabian Tronicke}
\chead{Gruppe: swp15.gkp}
%\cfoot{center of the footer!}
\renewcommand{\headrulewidth}{0.8pt}
%\renewcommand{\footrulewidth}{0.4pt}

\begin{document}
\center \large Universität Leipzig - Softwaretechnik Praktikum 2014/2015\\
\center \Huge Entwurfsbeschreibung\\
\par\bigskip

\small zum Projekt: Ein kartenbasiertes “Multiplayer”-Spiel

\par\bigskip

\tableofcontents

\clearpage

\flushleft
\section{Allgemeines}
\clearpage
\section{Produktübersicht}
\clearpage
\section{Grundsätzliche Struktur- und Entwurfsprinzipien}
\clearpage
\section{Struktur- und Entwurfsprinzipien der einzelnen Pakete}
\clearpage
\section{Datenmodell}
\clearpage
\section{Testkonzept}
\clearpage
\section{Glossar}
\clearpage
\end{document}