\documentclass[11pt,a4paper]{article}
%\usepackage{beamerarticle}

%\usefonttheme[onlymath]{serif}
\usepackage[ngerman]{babel}
\usepackage[utf8]{inputenc}
\usepackage[T1]{fontenc}
\usepackage{tikz}
\usetikzlibrary{positioning, arrows}
\usepackage{listings}
\usepackage{fancybox}
\usepackage{color}
\usepackage{hyperref}
\usepackage{fancyhdr}

\pagestyle{fancy}
\lhead{\today}
\author{Gruppe: SWT15-GKP}
\rhead{Author: PP}
\chead{Gruppe: SWT15-GKP}
%\cfoot{center of the footer!}
\renewcommand{\headrulewidth}{0.8pt}
%\renewcommand{\footrulewidth}{0.4pt}
\title{Universität Leipzig - Softwaretechnik Praktikum 2014/2015 \\  Storyplan \\ zum Projekt: Ein kartenbasiertes “Multiplayer”-Spiel}

\begin{document}
\maketitle

\clearpage

\flushleft


\section*{Woche 1}
\subsection*{Gruppe:PP, JR und GD} \par\bigskip
Aufgaben:
\begin{itemize}
\item /LF41/ Als Spieler möchte ich, dass die Levelerstellung deterministisch abläuft. 18
\item /LF11/ Als Spieler möchte ich entscheiden können an welchem Ort ich spielen möchte.
\item /LF23/ Als Akteur möchte ich mich vor dem Spiel über die Karte bewegen können.
\item /LF51/ Als Spieler möchte ich, dass während des Spielens die Karte nicht mehr verändert werden kann. 
\end{itemize}


\subsection*{Gruppe:JH, WB} \par\bigskip
Aufgaben:
\begin{itemize}
\item /LF15/ Als Spieler möchte ich über verbleibende Leben und Punkte per Headup informiert werden.
\item /LF52/ Als Spieler will ich, dass Pacman angezeigt und animiert wird.
\end{itemize}

\subsection*{Gruppe:KV, FT} \par\bigskip
Aufgaben:
\begin{itemize}
\item /LF30/ Als Spieler möchte ich sehen wer auf welcher Karte welche Highscores
erzielt hat. 16
\item /LF17/ Als Spieler möchte ich sowohl Hintergrundmusik als auch Soundeffekte hören.
\end{itemize}

\section*{Woche 2}

\subsection*{Gruppe:PP, GD}
Aufgaben:
\begin{itemize}
\item /LF41/ Als Spieler möchte ich, dass die Levelerstellung deterministisch abläuft. 7/18
\end{itemize} 
\subsection*{Gruppe:JH KV}
Aufgaben:
\begin{itemize}
\item /LF41/ Als Spieler möchte ich, dass die Levelerstellung deterministisch abläuft. 4/18
\item /LF43/ Als Betreuer möchte ich, dass die Kartengenese als Modul ausge-
führt ist. 4
\end{itemize}
\subsection*{Gruppe:JR, WB und FT}
Aufgaben:
\begin{itemize}
\item /LF41/ Als Spieler möchte ich, dass die Levelerstellung deterministisch abläuft. 7/18
\item /LF21/ Als Spieler möchte ich Pacman mit der Tastatur steuern können. 1
\end{itemize} 


\section*{Woche 3}
\subsection*{Gruppe:PP, JH und JR}
Aufgaben:
\begin{itemize}
\item /LF41/ Als Spieler möchte ich, dass die Levelerstellung deterministisch abläuft.  
\end{itemize}

\subsection*{Gruppe:KV, WB}
Aufgaben:
\begin{itemize}
\item  /LF41/ Als Spieler möchte ich, dass die Levelerstellung deterministisch abläuft.

\end{itemize}
\subsection*{Gruppe:GD, FT}
Aufgaben:
\begin{itemize}
\item /LF41/ Als Spieler möchte ich, dass die Levelerstellung deterministisch abläuft. 
\end{itemize}



\section*{Woche 4}

\subsection*{Gruppe:PP, JH}
Aufgaben:
\begin{itemize}
\item /LF53/ Als Spieler will ich, dass die Geister angezeigt und animiert werden. 8
\item /LF22/ Als Spieler möchte ich, dass die Geister verschiedene Strategien verfolgen. 12
\end{itemize}
\subsection*{Gruppe:JR, KV und GD}
Aufgaben:
\begin{itemize}
\item /LF41/ Als Spieler möchte ich, dass die Levelerstellung deterministisch abläuft.
\end{itemize}

\subsection*{Gruppe:WB, FT}
Aufgaben:
\begin{itemize}
\item /LF41/ Als Spieler möchte ich, dass die Levelerstellung deterministisch abläuft.
\end{itemize}
\section*{Woche 5}

\subsection*{Gruppe:PP, WB}
\begin{itemize}
\item /LF41/ Als Spieler möchte ich, dass die Levelerstellung deterministisch abläuft.
\end{itemize}
\subsection*{Gruppe:JH, FT}
\begin{itemize}
\item /LF41/ Als Spieler möchte ich, dass die Levelerstellung deterministisch abläuft.
\end{itemize}

\subsection*{Gruppe:JR, GD und KV}
\begin{itemize}
\item /LF41/ Als Spieler möchte ich, dass die Levelerstellung deterministisch abläuft.
\end{itemize}

\section*{Woche 6}
Debuggen, Dokumente vervollständigen
\end{document}