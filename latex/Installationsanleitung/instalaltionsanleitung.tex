\documentclass[11pt,a4paper]{article}
%\usepackage{beamerarticle}

%\usefonttheme[onlymath]{serif}
\usepackage[ngerman]{babel}
\usepackage[utf8]{inputenc}
\usepackage[T1]{fontenc}
\usepackage{tikz}
\usetikzlibrary{positioning, arrows}
\usepackage{listings}
\usepackage{fancybox}
\usepackage{color}
\usepackage{hyperref}
\usepackage{fancyhdr}
\usepackage{extarrows}

\pagestyle{fancy}
\lhead{\today}
\author{Gruppe: SWT15-GKP}
\rhead{Author: FT}
\chead{Gruppe: SWT15-GKP}
%\cfoot{center of the footer!}
\renewcommand{\headrulewidth}{0.8pt}
%\renewcommand{\footrulewidth}{0.4pt}
\title{Universität Leipzig - Softwaretechnik Praktikum 2014/2015 \\  Installationsanleitung \\ zum Projekt: Ein kartenbasiertes “Multiplayer”-Spiel}

\begin{document}
\maketitle

\tableofcontents

\clearpage

\section{Lokale Installation}
\subsection{Git installieren}
Um das Repository mit den Daten zu klonen, muss zu erst die Versionsverwaltungssoftware git installiert werden.
\subsubsection{Linux}
Installation über den Paketmanager, welcher vom Betriebssystem bereitgestellt wird.
Auf einem Debian basierten System ist dazu folgender Befehl ausführen:
\begin{lstlisting}
sudo apt-get install git
\end{lstlisting}
oder zum Beispiel unter Fedora:
\begin{lstlisting}
yum install git-core
\end{lstlisting}
\subsubsection{Windows}
Herunterladen und ausführen des Installationsprogramms von:
\begin{lstlisting}
http://msysgit.github.com/
\end{lstlisting}
Damit erhält man eine graphische Benutzeroberfläche, sowie eine Kommandozeilenversion.
Um die weiteren git-Befehle auszuführen, sollte die in msysGit enthaltene Shell verwendet werden.
\subsection{Repository klonen}
Um das Repository zu klonen ist folgender Befehl auszuführen:
\begin{lstlisting}
git clone git://github.com/GKP15/pucman.git
\end{lstlisting}
Die Daten befinden sich nun im Verzeichnis Pucman.
\subsection{Virtuoso installieren}

\section{Installation auf einem Webserver}
Die ersten Schritte entsprechen der Anleitung zu Instalaltion der lokalen Version:
\subsection{Git installieren}
Um das Repository mit den Daten zu klonen, muss zu erst die Versionsverwaltungssoftware git installiert werden.
\subsubsection{Linux}
Installation über den Paketmanager, welcher vom Betriebssystem bereitgestellt wird.
Auf einem Debian basierten System ist dazu folgender Befehl ausführen:
\begin{lstlisting}
sudo apt-get install git
\end{lstlisting}
oder zum Beispiel unter Fedora:
\begin{lstlisting}
yum install git-core
\end{lstlisting}
\subsubsection{Windows}
Herunterladen und ausführen des Installationsprogramms von:
\begin{lstlisting}
http://msysgit.github.com/
\end{lstlisting}
Damit erhält man eine graphische Benutzeroberfläche, sowie eine Kommandozeilenversion.
Um die weiteren git-Befehle auszuführen, sollte die in msysGit enthaltene Shell verwendet werden.
\subsection{Repository klonen}
Um das Repository zu klonen ist folgender Befehl auszuführen:
\begin{lstlisting}
git clone git://github.com/GKP15/pucman.git
\end{lstlisting}
Die Daten befinden sich nun im Verzeichnis Pucman.
\subsection{Virtuoso installieren}
Aus dem Olat, technisches Wiki:
Für jede Virtuoso-Instanz ist ein eigener Server aufzusetzen, der auf gemeinsame Binaries der Virtuoso Open Source Edition zugreift. Die Datenbasis (die den Zustand Ihrer Virtuoso-Instanz persistiert und kapselt) und weitere zum Betrieb erforderliche Dateien werden dazu in einem Verzeichnis Ihrer Wahl verwaltet, das Sie dazu anlegen und vorbereiten müssen.
Entpacken Sie dazu die angehängte Datei virtuoso.zip. Es entsteht ein Verzeichnis virtuoso mit den Dateien start.sh und virtuoso.ini, die weiter anzupassen sind:

\begin{enumerate}
	\item Ersetzen Sie in beiden Dateien /Pfad/zu/ durch den absoluten Pfad Ihrer Installation.
	\item Nutzen Sie einen der oben gelisteten freien Ports (und tragen Sie das Projekt dort ein), unter dem Ihr Virtuoso-Server laufen wird. Der erste Eintrag [NameDerSektion] ist der Name der Sektion in die Datei /etc/odbc.ini, über die Ihre Anwendungen auf Virtuoso zugreifen kann (wenn erforderlich, etwa OntoWiki oder xodx). Der zweite Eintrag (SP) ist der V-Serverport. Ersetzen Sie in der Datei virtuoso.ini den Port 1111 durch diesen V-Serverport. Der dritte Eintrag (HSP) ist der HTTPS-Port des zugehörigen Virtuoso Sparql Endpunkts. Tragen Sie diesen in der Datei virtuoso.ini statt 8890 ein.
\end{enumerate}
Starten Sie den Server mit
\begin{lstlisting}
sh start.sh
\end{lstlisting}
Greifen Sie mit der Kommandozeilenversion
\begin{lstlisting}
isql-vt <Ihr-V-Port> dba dba
\end{lstlisting}
auf die neu angelegte Instanz zu und ändern zunächst das Default Passwort 'dba':
\begin{lstlisting}
SQL> set password dba YourVerySecretPassword ;
\end{lstlisting}
Der Server wird mit
\begin{lstlisting}
SQL> shutdown() ;
\end{lstlisting}
heruntergefahren und mit
\begin{lstlisting}
isql-vt <Ihr-V-Port> dba YourVerySecretPassword
\end{lstlisting}
wieder hochgefahren.
\subsection{Auf Server laden}

\end{document}