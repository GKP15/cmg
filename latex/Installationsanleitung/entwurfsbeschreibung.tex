\documentclass[11pt,a4paper]{article}
%\usepackage{beamerarticle}

%\usefonttheme[onlymath]{serif}
\usepackage[ngerman]{babel}
\usepackage[utf8]{inputenc}
\usepackage[T1]{fontenc}
\usepackage{tikz}
\usetikzlibrary{positioning, arrows}
\usepackage{listings}
\usepackage{fancybox}
\usepackage{color}
\usepackage{hyperref}
\usepackage{fancyhdr}
\usepackage{extarrows}

\pagestyle{fancy}
\lhead{\today}
\author{Gruppe: SWT15-GKP}
\rhead{Author: FT}
\chead{Gruppe: SWT15-GKP}
%\cfoot{center of the footer!}
\renewcommand{\headrulewidth}{0.8pt}
%\renewcommand{\footrulewidth}{0.4pt}
\title{Universität Leipzig - Softwaretechnik Praktikum 2014/2015 \\  Installationsanleitung \\ zum Projekt: Ein kartenbasiertes “Multiplayer”-Spiel}

\begin{document}
\maketitle

\tableofcontents

\clearpage

\section{lokale Installation}
\subsection{Git installieren}
\subsection{Npm installieren}
\subsection{Datenbankzeugs installieren}

\section{Installation auf einem Webserver}
\subsection{Git installieren}
\subsection{Auf Server laden}
\subsection{Datenbankzeugs installieren}

\end{document}